% !TEX root=report.tex
\section{Introduction} \label{sec:introduction}

\begin{quotation}
Among those to whom I did not write ``couch request'' were ``a travelling magician and professional fool'' from New Mexico; a sixty-three-year-old gay semi-retired handyman in Pahoa, Hawaii, whose mission is ``looking for more nudists'' (there are plenty of ``clothing optional'' possibilities on CouchSurfing); another Hawaiian, this one describing himself as ``just a guy who has three acres of land, living in a shipping container house''; a woman in Bozeman, Montana, who declared that her ``home is oppression-free. Yay!'' and also contains high-speed Wi-Fi; a thirty-one-year-old female ``daydreamer'' from Berkeley who loves pajamas; a Tarot-card reader in Marfa, Texas; a housewife in Charleston, South Carolina, who owns a pole-dancing studio called Dolphin Dance; a kite-surfing physician (again, Charleston) whose ambition is to start a flavored-envelope company; a freelance photographer in San Francisco who claims she’s had hiccups every day for the past five years; a Savannah, Georgia, scientist who is also a ``free man,'' doing ``whatever I want, whenever I want''; a male in Antarctica who wishes to live as a female; a ``lovetarian'' who grew up in Doylestown, Pennsylvania; ... and eighty-four people in Brunei Darussalam.

-- Patricia Marx. \emph{You're Welcome: Couch-surfing the globe}. New Yorker, April 16, 2012.\footnote{Retrieved from \url{http://www.newyorker.com/reporting/2012/04/16/120416fa_fact_marx}}
\end{quotation}

Our new global interconnectedness has made social interactions that were previously unthinkable commonplace.
Even a couple of decades ago a traveler had to either have a friend or pay for a hotel for lodging while traveling, but today people so inclined can stay for free at complete strangers' residences, in locations that span all countries and continents.
Such accomodations are arranged via special-purpose websites that let users willing to host travelers to list their apartments (``hosts''), and users wanting to with someone for free to request lodging from the hosts (``surfers'').
Among these websites, by far the biggest is CouchSurfing (\url{http://couchsurfing.org/}), a website launched in 2004 and now boasting almost four million users.

There is a staggering variety of people on CouchSurfing.
There are members in every country, speaking over three hundred different languages.
The age distribution skews young: the average age is twenty-eight and a third are between 18 and 24.
The website has facilitated over six million positively-reviewed visits, with ``only a tiny fraction of one per cent negative.''\footnote{from the article quoted in the epigraph.}

A fraternizing spirit is strong with most users of the site, as they are willing to host or guest with strangers in exchange for no money.
All the same, we believe that people there are some user-match relationships that are more likely to be successful than others.
Hosts are selective in accepting requests, and it should be possible to learn from their preferences.

When a surfer is looking for lodging, they do a search for available couches (or beds, or the floor, as the case may be) in the destination location.
Usually, they are presented with an array of choices, often more than can quickly be looked through---or messaged.
It would be beneficial to a searching surfer to have hosts that are available, agreeable, and likely to respond positively displayed at the top of the list.
Similarly, hosts in popular locations are inundated with requests for accomodation.
To save time, surfers that would be agreeable to them should be displayed at the top.
An example of a host's inbox is presented in \autoref{fig:sample_request}.

In this work, we develop a system to automatically evaluate the couch requests that hosts get with respect to the likelihood of a successful and pleasant visit, and so make the CouchSurfing experience as convenient as possible.
While we train and evaluate our approach from the perspective of the hosts, our solution to the problem---predicting the probability of a host accepting a surfer request---benefits surfers just as much, as their search results can be ranked according to the likelihood of acceptance to save them message-writing effort.

We detail our problem formulation further while evaluating related work in \autoref{sec:related_work}.
Next, we look at the structure of CouchSurfing data and user experience in \autoref{sec:data}.
In \autoref{sec:model}, we formalize the problem and present our model. Our approach to extracting features from user data is presented in \autoref{sec:features}.
Finally, evaluation results are presented in \autoref{sec:evaluation}, and conclusions drawn in \autoref{sec:conclusion}.

\begin{figure}[ht]
\centering
\includegraphics[width=0.8\linewidth]{figures/screenshots/requests.png}
\caption{Example of a host's inbox view.}
\label{fig:sample_request}
\end{figure}

