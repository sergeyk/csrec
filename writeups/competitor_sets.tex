% !TEX root=report.tex
\subsection{Competitor Sets} \label{subsec:competitor_sets}

The data we received from CouchSurfing does not allow us to know exactly when a host has looked at a couch request, but it does have timestamps for every request and every decision made by a host.
To form sets of ``competing'' requests, we use a heuristic procedure to scan through all requests made to a particular host, and split them into groups based on their timing.

In the following, requests carry the same index as the user/surfer that invoked them, i.e., surfer $s_i$ sends request $r_{i,j}$ to host $h_j$.
In the following examples, $h$ is fixed, so we just denote requests as $r_i$.

We assume a session-based behavior for the user $h$, which means he sits down in front of his computer and checks all new request to his couch every once in a while. There will be several different surfers that want to stay with him during different times. Now within a certain concetration span, $h$ will read through all the requests (besides the basic infos about age, sex and location, a potential surfer $s_i$ can also specify various details about his itinerary and whatever else will help him convince $h$). While reading through the requests $h$ decides on the fly which of them to accept or decline. Based on this notion we chunk together request by their \textit{request modification date} (rmd), the date $h$ responded uppon them with \textit{accept}, \textit{declined} or \textit{maybe}. More specifically, we order all request for each host by their rmd and perform a mean-shift clustering with a bandwidth of the session length, e.g. 30 min.


\begin{figure}[ht]
\centering
\subfloat[Requested]{
  \includegraphics[width=0.45\linewidth]{figures/top_requested.png}
}
\subfloat[Accepted]{
  \includegraphics[width=0.45\linewidth]{figures/top_accepted.png}
}
\caption{\todo{for tobi: Explain.}}
\label{fig:timeline_view}
\end{figure}
